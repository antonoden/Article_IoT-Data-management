\documentclass[article,a4paper]{IEEEtran}
\usepackage{lipsum}
\usepackage[backend=biber]{biblatex}
\usepackage{graphicx}

\addbibresource{refs.bib}
\title{Syncronization time, JSON, }
\author{
\IEEEauthorblockN{Anton Odén}\\
\IEEEauthorblockA{Dept. of Maths and Computer Science\\Karlstad University\\
651 88 KARLSTAD, Sweden}\\
anton.oden@outlook.com
}

\begin{document}

\maketitle

\section{Introduction}
The goal of IoT is to give us data to be able to make informed decisions and let us make decisions that would otherwise never been made but because of IoT is being made and those decisions are contributing to the greater good. In a democratic socitity in a reccurent event every citizen with the right to vote is asked to cast a vote on who should make decisions for the next period. The votes are cast in a ballot box and the votes are counted. All the apparatus around the vote and counting it is a costly operation and in Sweden the authorities of election estimated an reelection to cost 200 million swedish crowns \cite{costElection}. We do it nevertheless because we believe it is an important decision to be made. Thought not every opinion of the citizen has been taken into consideration in that single vote that would give the govermental aparatues the informed knowledge to rule. Alot of other opinions from the citizens would also be beneficial to have gotten voted on to make informed decisions that would otherwise never been made, or that are made but made wrong becuase of lack of information/opinion. Our IoT nodes are also citizens. They are supplying data that could and should help in making informed decisions and just as in politics there is decisions to be made in IoT how close to the end nodes the decision making is to be made. Should all data be communicated to the cloud to be processed there or should the data be processed closer to the end nodes? The answer to that question is not easy and there are many factors to take into consideration. One of the factors is the time it takes to communicate the data to the cloud and back. Another is the bandwidth of the communication channel taken up when all things communicate on it. The highways has to be broadened to the able to take the load and there is costs associated with that. This article will...
\section{Background}  
JavaScript Object Notation (JSON) is the format of the payload of our communication. It doesn't have to be JSON but it has become a standard and is used in many applications. It is commonly used in data transmission between websites and it its where its orgininate from but it has become an an internatinal data processing standard since 2013. It is lightweight and easy for humans to read. JSON is a textformat that is completely language independent and is used in many programming languages.      
\printbibliography

\end{document}